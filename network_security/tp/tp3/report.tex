\documentclass[a4paper,10pt,final,fleqn]{article}
\usepackage[frenchb]{babel}
\usepackage{fontenc}
\usepackage{fancyhdr} % Required for custom headers
\usepackage{lastpage} % Required to determine the last page for the footer
\usepackage{extramarks} % Required for headers and footers
\usepackage{graphicx} % Required to insert images
\usepackage[utf8]{inputenc}
\usepackage{apacite}
\usepackage{url}
\usepackage[normalem]{ulem}
\usepackage{verbatim}
\usepackage{hyperref}

\evensidemargin=0in
\oddsidemargin=0in
\textwidth=6in
\textheight=9.0in
\headsep=0.25in 

% Set up the header and footer
\pagestyle{fancy}
\lhead{\GroupeName} % Top left header
\chead{\CourseAndAPP} % Top center header
\rhead{\Date} % Top right header
\lfoot{\lastxmark} % Bottom left footer
\cfoot{} % Bottom center footer
\rfoot{\ \thepage\ / 	\pageref{LastPage}} % Bottom right footer
\renewcommand\headrulewidth{0.4pt} % Size of the header rule
\renewcommand\footrulewidth{0.4pt} % Size of the footer rule

\setcounter{tocdepth}{2}

\newcommand{\GroupeName}{4.6}
\newcommand{\CourseAndAPP}{NetSec} % Course/APP
\newcommand{\Date}{24 Septembre 2015}

\title{
\parbox{15cm}
{ %\includegraphics[width=4cm]{fox.png} \\
  \vspace{3cm}
	\begin{center}\sf\bfseries\Huge
		\rule{15cm}{1pt}
		\medskip
		Sécurité des Réseaux - TP3 \\
		\huge Contrôle d'Accès
		\vspace{.5cm}
		\rule{15cm}{1pt}
	\end{center}
	\vspace{3cm}
 }} 
\author{Monroe Samuel}
\date{\today}

\begin{document}
\maketitle
\newpage

	\section{Préambule}

		Ce tp est réalisé avec CentOS Security virtualisé de l'EPHEC avec VMWare.

	\section{Contrôle d'accès aux fichiers Unix} 

		\subsection{Utilisation du contrôle d'accès}

			\subsubsection{Création des users et groupes}

				La création des utilisateurs se fait via la commande \textbf{adduser nom\_du\_user}\\

				La création des groupes se fait via la commande \textbf{groupadd nom\_du\_groupe}\\

				La commande \textbf{usermod} permet d'ajouter un utilisateur existant à un groupe, ici utilisée de la manière suivante : \textbf{usermod -a -G grp1 userX}\\
			
			\subsubsection{/etc/group}

				Le fichier contient des entrées du type : \textbf{kek:x:502:user1,user2}\\

				Ces entrées ont donc les champs suivants : \\

				\begin{itemize}
					\item Le nom du groupe
					\item Un mot de passe, souvent non utilisé et laissé blank (x)
					\item Le GID, group id
					\item La liste des users appartenant à ce groupe
				\end{itemize}

			\subsubsection{Différences entre /etc/passwd, /etc/shadow}

				La fichier \textbf{passwd} associe à chaque entrée, qui correspond un user, le nom complet du User et le chemin absolu vers son \textbf{home} terminé par son compte shell.\\

				Le fichier \textbf{shadow} quant à lui, contient : \\

				\begin{itemize}
					\item Le nom du user
					\item Le mot de passe codé
					\item Des nombres de jours à propos des modifications de mots de passes (update, validité, expiration,)
				\end{itemize}

			\subsubsection{Accès au répertoire}
		
				Le répertoire de userX a tout les droits au niveau user et groupe, les autres ont le droit à X et R mais pas en écriture.\\

				Un test avec userZ, appartenant au groupe 2, n'est pas possible étant donné que le répertoire Home ne propose des droits qu'au propriétaire ! \\

			\subsubsection{Cascade des droits}

				Afin de propager les droits, le bits SGID doit être positionné afin que les fichiers et répertoires suivants héritent de ces mêmes droits.\\

				\textbf{chmod 2xxxx dir}\\

			\subsubsection{Suppression des fichiers personnels}

				Ceci est effectué en positonnant le sticky bit, de sorte qu'un user ne puisse supprimer que les fichiers qui lui appartiennent dans ce répertoire.\\

				\textbf{chmod 1xxx dir}\\

			\subsubsection{Droits d'accès fichiers spéciaux}

				 \begin{itemize}
				 	\item /dev/sda : brw-rw----
				 	\item /dev/tty : crw-rw-rw
				 	\item /dev/lp  : crw-rw----
				 \end{itemize}

			\subsubsection{Droits défault dans home}

				Les droits par défaut sont RWX pour le propriétaire uniquement.\\

\end{document}
