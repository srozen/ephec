\documentclass[a4paper,10pt,final,fleqn]{article}
\usepackage[frenchb]{babel}
\usepackage{fontenc}
\usepackage{fancyhdr} % Required for custom headers
\usepackage{lastpage} % Required to determine the last page for the footer
\usepackage{extramarks} % Required for headers and footers
\usepackage{graphicx} % Required to insert images
\usepackage[utf8]{inputenc}
\usepackage{apacite}
\usepackage{url}
\usepackage[normalem]{ulem}
\usepackage{verbatim}
\usepackage{listings}

\evensidemargin=0in
\oddsidemargin=0in
\textwidth=6in
\textheight=9.0in
\headsep=0.25in 

% Set up the header and footer
\pagestyle{fancy}
\lhead{\GroupeName} % Top left header
\chead{\CourseAndAPP} % Top center header
\rhead{\Date} % Top right header
\lfoot{\lastxmark} % Bottom left footer
\cfoot{} % Bottom center footer
\rfoot{\ \thepage\ / 	\pageref{LastPage}} % Bottom right footer
\renewcommand\headrulewidth{0.4pt} % Size of the header rule
\renewcommand\footrulewidth{0.4pt} % Size of the footer rule

\setcounter{tocdepth}{2}

\newcommand{\GroupeName}{4.6}
\newcommand{\CourseAndAPP}{NetSec} % Course/APP
\newcommand{\Date}{Septembre 2015}

\title{
\parbox{15cm}
{ \includegraphics[width=4cm]{foxhound.png} \\
  \vspace{3cm}
	\begin{center}\sf\bfseries\Huge
		\rule{15cm}{1pt}
		\medskip
		Sécurité des Réseaux - TP \\
		\huge Iptables
		\vspace{.5cm}
		\rule{15cm}{1pt}
	\end{center}
	\vspace{3cm}
 }} 
\author{Monroe Samuel}
\date{\today}

\begin{document}
\maketitle
\newpage

	\section{Exercices de Théorie}

		\subsection{Exercice 1}

			Les trois chaînes ont une policy de DROP par défaut.\\

			\begin{itemize}
				\item INPUT : Accepte du ICMP peu importe les conditions
				\item FORWARD : Accepte de l'udp depuis 192.168.1.10 vers n'importe où, et accepte de l'udp depuis n'importe où vers 192.168.1.10
				\item OUTPUT : Pas de règle
			\end{itemize}

		\subsection{Exercice 2}

			Les commandes font un \textbf{flush} de toutes les règles précédentes, mettent une \textbf{policy} de drop par défaut sur toutes les chaînes, ajoutent sur \textbf{INPUT} l'acceptation de messages ICMP echo-request.\\
			Sur \textbf{OUTPUT}, accepte les messages ICMP

		\subsection{Exercice 3}

			La chaîne \textbf{PREROUTING} a une policy d'acceptation par défaut.\\
			La chaîne \textbf{POSTROUTING} a une policy d'acceptation par défault, et une règle de NAT qui modifie les adresses IP source du réseau local vers n'importe quelle destination.\\
			La chaîne \textbf{OUTPUT} a une policy d'acceptation par défaut.\\

		\subsection{Exercice 4}

			Flush de toutes les règles précédentes, et ajout d'une policy \textbf{DROP} par défaut sur les trois chaînes.\\
			Ajoute une règle d'acceptation en INPUT de nouvelles connexions ou de connexions établies sur les ports 80 et 443 (HTTP(S))\\
			En \textbf{OUTPUT}, règle d'acceptation de toute connexion en état établie.\\

		\subsection{Exercice 5}

			\begin{itemize}
				\item Flush de la chaîne \textbf{Forward}
				\item Ajout d'une règle d'acceptance sur forward
				\item Sur la table NAT, la chaîne \textbf{PREROUTING} est créditée d'une règle de NAT dynamique pour NAter les adresses du LAN.
			\end{itemize}

		\subsection{Exercice 6}

			On flush les trois chaînes de la table \textbf{FILTER}, et on leur ajoute une policy d'acceptation.\\
			On flush la chaînes \textbf{POSTROUTING} de la table \textbf{NAT}, et on ajoute une règle qui va modifier (\textbf{SNAT}) l'adresse source publique à une adresse publique dans un range de 10 adresses.\\

		\subsection{Exercice 7}

			On flush les trois chaînes de la table \textbf{FILTER}, et on leur ajoute une policy d'acceptation.\\
			On flush la chaînes \textbf{PREROUTING} de la table \textbf{NAT}, et on ajoute une règle qui va modifier (\textbf{DNAT}) l'adresse de destination publique à une adresse privée port 80.\\

		\subsection{Exercice 8}

			On ajoute des règles en INPUT et OUTPUT, pour l'interface ppp0 en sortie et en entrée, qui vont accepter le protocole DNS port 53 en TCP et en UDP.\\

		\subsection{Exercice 9}

			\begin{enumerate}
				\item Active le routage ipv4
				\item Flush de toutes les règles
				\item Création d'une chaîne personelle LOG\_DROP, on spécifie que les outputs de cette chaîne vont avoir un préfixe textuel et on spécifie également que tout paquet amené sur cette chaîne doit être droppé.
				\item Création d'une chaîne personelle LOG\_ACCEPT, on spécifie que les outputs de cette chaîne vont avoir un préfixe textuel et on spécifie également que tout paquet amené sur cette chaîne doit être accepté.
				\item Les policies sur les chaînes \textbf{INPUT, OUTPUT, FORWARD} sont mise à \textbf{DROP}.
				\item Sur la chaîne \textbf{INPUT}, l'interface l0 en input est mise à \textbf{ACCEPT}, et sur la chaîne \textbf{OUTPUT}, l'interface l0 en output est mise à \textbf{ACCEPT}
				\item Sur la chaîne \textbf{FORWARD}, ACCEPT en input sur la eth1 et output sur ppp0, et inversément.\\
				De plus, sur la tâble nat chaîne \textbf{POSTROUTING}, une règle NAT les adresses du réseaux local.\\
				\item Sur la chaîne \textbf{OUTPUT}, on accepte en sortie de l'interface ppp0, les connexions en état nouveau ou établie sur le port 80.\\
				Sur la chaîne \textbf{INPUT}; on accepte en entrée de l'interface ppp0 les connexion établies sur le port 80.\\
				Enfin, sur la tâble NAT chaîne \textbf{PREROUTING}, la règle NATe l'adresse destination des paquets en entrée de l'interface eth1 sur le port 80 vers l'adresse 192.168.2.1 port 3128.\\
				\item Sur la tâble NAT chaîne \textbf{PREROUTING}, la règle NATe l'adresse destination 42.42.42.42 port 80 vers l'adresse 192.168.1.2 port 80.\\
				Sur la chaîne \textbf{FORWARD}, on accepte les paquets entrants sur ppp0 à destination du port 80 pour des connexions NEW ou ESTABLISHED.\\
				Sur la chaîne \textbf{FORWARD}, on accepte les paquets sortants sur ppp0 à destination du port 80 pour des connexions ESTABLISHED.\\
				Sur la table \textbf{NAT}, chaîne \textbf{POSTROUTING} on NATe les paquets du LAN.\\
				\item Sur la chaîne \textbf{FORWARD}, on accepte les paquets entrants sur eth1 pour le port 80 entrant pour des connexions nouvelles ou établies.\\
				On accepte également les paquets entrants sur eth1 pour le port 80 sortant pour des connexions établies.\\
				\item Sur la chaîne \textbf{INPUT}, on transfère à la liste LOG\_ACCEPT les paquets entrants sur eth1, depuis 192.168.2.42 à destination du port 22 pour des nouvelles connexions ou établies.\\
				Sur la chaîne \textbf{OUTPUT}, on transfère à la liste LOG\_ACCEPT les paquets sortants sur eth0, vers l'adresse 192.168.2.42 port 200, pour des connexions établies.\\
				\item On jump les trois chaînes de la table \textbf{filter} sur LOG\_DROP.
			\end{enumerate}

	\section{Laboratoire}

		\begin{enumerate}
			\item Les trois commandes suivantes ajoutent une policy d'acceptation pour tout paquet : \\

				\begin{itemize}
					\item iptables -P INPUT ACCEPT
					\item iptables -P OUTPUT ACCEPT
					\item iptables -P FORWARD ACCEPT
				\end{itemize}

			\item Rejet de tout le traffic, et logging des outputs, les commandes fonctionnent et on observe bien les rejets dans /var/log/messages \\

				\begin{itemize}
					\item iptables --flush (flush de toutes les règles)
					\item iptables -N LOGS (création d'une liste custom LOGS)
					\item iptables -P INPUT DROP
					\item iptables -P OUTPUT DROP
					\item iptables -P FORWARD DROP
					\item iptables -A INPUT -j LOGS (jump da la liste dans logs)
					\item iptables -A OUTPUT -j LOGS	
					\item iptables -A FORWARD -j LOGS
					\item iptables -A LOGS -m limit -j LOG --log-prefix "IPTables-Dropped: " --log-level 4
					\item iptables -A LOGS -j DROP
				\end{itemize}

			\item Remise du firewall dans sa config par défault, configurer le firewall pour empêcher toute requête html vers pc3 qui possède un site web. \\

			Le serveur web du PC3 affiche une page html avec "It works ! ".\\

				\begin{itemize}
					\item iptables -A INPUT -p tcp --destination-port 80 -j DROP
					\item iptables -A OUTPUT -p tcp --destination-port 80 -j DROP
					\item iptables -A FORWARD -p tcp --destination-port 80 -j DROP
				\end{itemize}

			\item Dans l'exercice précédent, la policy par défaut était ACCEPT donc je garde ici la configuration précédente.\\
			Cette commande permet toujours le ping par PC4, mais le subnet précisé ne peut plus pinguer PC3.\\

				\begin{itemize}
					\item iptables -A FORWARD -p icmp -s 192.168.2.0/24 -d 192.168.1.3 -j DROP
				\end{itemize}

			\item Ici encore, je garde la configuration de policy d'acceptance par défaut, pour bien voir que les échecs et l'acceptance sont dûs à mes règles.\\

			Je précise d'abord une règle d'accès pour le PC2 sur l'interface du côté de PC2, ensuite un refus pour tout autre source.\\
			Ensuite on rejette le reste dans la liste LOGS, qui elle-même rejera un jump sur drop.\\

				\begin{itemize}
					\item Flush puis acceptance par défaut.
					\item iptables -A INPUT -i eth0 -p tcp --destination-port 22 -s 192.168.2.2/24 -d 192.168.2.200/24 -j ACCEPT
					\item iptables -A OUTPUT -p tcp --source-port 22 -d 192.168.2.2/24 -j ACCEPT
					\item iptables -A INPUT -j LOGS
					\item iptables -A OUTPUT -j LOGS
					\item iptables -A LOGS -m limit -j LOG --log-prefix "IPTables-Dropped: " --log-level 4
					\item iptables -A LOGS -j DROP
				\end{itemize}
		\end{enumerate}

\end{document}