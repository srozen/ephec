\documentclass[a4paper,10pt,final,fleqn]{article}
\usepackage[frenchb]{babel}
\usepackage{fontenc}
\usepackage{fancyhdr} % Required for custom headers
\usepackage{lastpage} % Required to determine the last page for the footer
\usepackage{extramarks} % Required for headers and footers
\usepackage{graphicx} % Required to insert images
\usepackage[utf8]{inputenc}
\usepackage{apacite}
\usepackage{url}
\usepackage[normalem]{ulem}
\usepackage{verbatim}
\usepackage{hyperref}
\usepackage{listings}

\evensidemargin=0in
\oddsidemargin=0in
\textwidth=6in
\textheight=9.0in
\headsep=0.25in 

% Set up the header and footer
\pagestyle{fancy}
\lhead{\GroupeName} % Top left header
\chead{\CourseAndAPP} % Top center header
\rhead{\Date} % Top right header
\lfoot{\lastxmark} % Bottom left footer
\cfoot{} % Bottom center footer
\rfoot{\ \thepage\ / 	\pageref{LastPage}} % Bottom right footer
\renewcommand\headrulewidth{0.4pt} % Size of the header rule
\renewcommand\footrulewidth{0.4pt} % Size of the footer rule

\setcounter{tocdepth}{2}

\newcommand{\GroupeName}{4.6}
\newcommand{\CourseAndAPP}{NetSec} % Course/APP
\newcommand{\Date}{24 Septembre 2015}

\title{
\parbox{15cm}
{ %\includegraphics[width=4cm]{fox.png} \\
  \vspace{3cm}
	\begin{center}\sf\bfseries\Huge
		\rule{15cm}{1pt}
		\medskip
		Sécurité des Réseaux - TP4 \\
		\huge Injections SQL
		\vspace{.5cm}
		\rule{15cm}{1pt}
	\end{center}
	\vspace{3cm}
 }} 
\author{Monroe Samuel}
\date{\today}

\begin{document}
\maketitle
\newpage

	\section{Préambule}

		Ce tp est réalisé avec l'image Debian 2.6.32 fournie sur le site de Pentesterlab.

	\section{Fingerprinting}

		Le hostname ne fonctionnant pas, j'ai lancé le telnet sur 127.0.0.1 port 80.\\

		La requête \textbf{GET} indique effectivement que le serveur tourne sous Apache 2.2.6 avec PHP 5.3.3.\\

	\section{Détection et exploitation d'injections SQL}

		\subsection{Détection sur les entiers}

			La manipulation des entiers dans l'url peut mettre en lumière une faille sur la façon dont ils sont traités.\\
			Dans ce cas-ci, on remarque qu'en tappant \textbf{?id=2-1} dans l'url, PHP traite cette valeur comme une soustraction et nous affiche la page correspondant à \textbf{?id=1}.\\

			
		
		