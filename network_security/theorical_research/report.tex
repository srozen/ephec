\documentclass[a4paper,10pt,final,fleqn]{article}
\usepackage[frenchb]{babel}
\usepackage{fontenc}
\usepackage{fancyhdr} % Required for custom headers
\usepackage{lastpage} % Required to determine the last page for the footer
\usepackage{extramarks} % Required for headers and footers
\usepackage{graphicx} % Required to insert images
\usepackage[utf8]{inputenc}
\usepackage{apacite}
\usepackage{url}
\usepackage[normalem]{ulem}
\usepackage{verbatim}
\usepackage{hyperref}
\usepackage{listings}

\evensidemargin=0in
\oddsidemargin=0in
\textwidth=6in
\textheight=9.0in
\headsep=0.25in 

% Set up the header and footer
\pagestyle{fancy}
\lhead{\GroupeName} % Top left header
\chead{\CourseAndAPP} % Top center header
\rhead{\Date} % Top right header
\lfoot{\lastxmark} % Bottom left footer
\cfoot{} % Bottom center footer
\rfoot{\ \thepage\ / 	\pageref{LastPage}} % Bottom right footer
\renewcommand\headrulewidth{0.4pt} % Size of the header rule
\renewcommand\footrulewidth{0.4pt} % Size of the footer rule

\setcounter{tocdepth}{2}

\newcommand{\GroupeName}{4.6}
\newcommand{\CourseAndAPP}{NetSec} % Course/APP
\newcommand{\Date}{24 Septembre 2015}

\title{
\parbox{15cm}
{ %\includegraphics[width=4cm]{fox.png} \\
  \vspace{3cm}
	\begin{center}\sf\bfseries\Huge
		\rule{15cm}{1pt}
		\medskip
		Sécurité des Réseaux \\
		\huge Hardening d'un VPS Web
		\vspace{.5cm}
		\rule{15cm}{1pt}
	\end{center}
	\vspace{3cm}
 }} 
\author{Monroe Samuel}
\date{\today}

\begin{document}
\maketitle
\newpage

	\section{Introduction}

		Ce document vous donnera une description du sujet de ma recherche théorique dans le cadre du cours de sécurité des réseaux, donné par Mme. Van Den Schrieck.\\

	\section{Description du Sujet}

		Ma recherche théorique portera sur le \textbf{hardening} d'un VPS.\\

		Ce VPS est un serveur virtuel hébergé dans le cloud par un fournisseur de services de ce type (DigitalOcean, OVH).\\
		Ce serveur tournera sous un OS Linux ou BDS, qui lui-même fera tourner un serveur Web hébergant un site web dynamique de vente en ligne de packs pour faire des hackathons chez soi. Il y aura donc également une base de données qui tournera sur ce serveur.\\

		Le VPS sera donc administré à distance par le chef de projet, et celui-ci étant étudiant, son projet sera versionné sur un repo GitHub privé.\\

\end{document}
