\documentclass[a4paper,10pt,final,fleqn]{article}
\usepackage[frenchb]{babel}
\usepackage{fontenc}
\usepackage{fancyhdr} % Required for custom headers
\usepackage{lastpage} % Required to determine the last page for the footer
\usepackage{extramarks} % Required for headers and footers
\usepackage{graphicx} % Required to insert images
\usepackage[utf8]{inputenc}
\usepackage{apacite}
\usepackage{url}
\usepackage[normalem]{ulem}
\usepackage{verbatim}
\usepackage{listings}

\evensidemargin=0in
\oddsidemargin=0in
\textwidth=6in
\textheight=9.0in
\headsep=0.25in

% Set up the header and footer
\pagestyle{fancy}
\lhead{\GroupeName} % Top left header
\chead{\CourseAndAPP} % Top center header
\rhead{\Date} % Top right header
\lfoot{\lastxmark} % Bottom left footer
\cfoot{} % Bottom center footer
\rfoot{\ \thepage\ / 	\pageref{LastPage}} % Bottom right footer
\renewcommand\headrulewidth{0.4pt} % Size of the header rule
\renewcommand\footrulewidth{0.4pt} % Size of the footer rule

\setcounter{tocdepth}{2}

\newcommand{\GroupeName}{HE201041}
\newcommand{\CourseAndAPP}{Projet Iptables} % Course/APP
\newcommand{\Date}{\today}

\title{
\parbox{15cm}
{ %\includegraphics[width=4cm]{foxhound.png} \\
  \vspace{3cm}
	\begin{center}\sf\bfseries\Huge
		\rule{15cm}{1pt}
		\medskip
		Sécurité des Réseaux \\
		\huge Validations Projet IpTables
		\vspace{.5cm}
		\rule{15cm}{1pt}
	\end{center}
	\vspace{3cm}
 }}
\author{Youri Monton, Victorien Derasse, Alexandre Bergiers, Monroe Samuel}
\date{\today}

\begin{document}
\maketitle
\thispagestyle{empty}
\mbox{}

\section{Procédure de validation}

	Ces procédures de validation répertorient les commandes entrées sur les différentes machines et le résultat obtenu, de manière manuelle et via netcad.\\

	Nous avons donc utilisé entre autre l’outil netcat (nc) permettant de tester les différents services  et vérifier les règles de sécurités utilisées dans le projet iptables. (Ex : nc –z 192.168.2.10). L’option z de netcat est utilisée pour scanner le résultat.\\

	\subsection{R1}

	\begin{itemize}
		\item Connexion en SSH directement sur Processor, puis connexion sur FTP
		\item Accès au web distant, accès au site de paranoyak
		\item Test d'envoi et de réception d'un mail
		\item NFS - Non fonctionnel suite à des soucis de règles iptables
		\item Rsync en déplaçant un fichier sur le serveur
		\item FTP via une connexion au serveur et un échange de fichiers\\
	\end{itemize}

	Voici également les commandes nc : \\

	ssh:          # nc -z 192.168.1.10 22\\
	rsync:       # nc -z 192.168.7.12 873\\
	ftp:          # nc -z 192.168.1.11 21\\
	ldns:        # nc -z 192.168.2.10 53\\
	smtp:       # nc -z 192.168.2.13 25\\
	imap:       # nc -z 192.168.2.13 143\\
	web:        # nc -z 192.168.7.10 80\\
	nfs:         # nc -z 192.168.1.12 2049\\
	pdns:       # nc -z 192.168.7.11 53\\
	processor: # nc -z 192.168.4.10 22\\

	\subsection{R2}

	\begin{itemize}
		\item Idem que pour R2 sauf la connexion directe à processor
		\item COnnexion en SSH à SSH et test d'autres services
	\end{itemize}


	Commandes Netcad : \\

	ssh:    # nc -z 192.168.1.10 22\\
	rsync: # nc -z 192.168.1.12 873\\
	ftp:    # nc -z 192.168.1.11 21\\
	ldns:  # nc -z 192.168.2.10 53\\
	smtp: # nc -z 192.168.2.13 25\\

	\subsection{U1 et U2}

		\begin{itemize}
			\item Accès au web
			\item Echange de mails
		\end{itemize}

	\subsection{Processor}

		\begin{itemize}
			\item Accès au ftp pour le téléchargement de fichiers
		\end{itemize}

	\subsection{T1}

		\begin{itemize}
			\item Accès aux services web, pdns, rsync et mail.
		\end{itemize}

\end{document}