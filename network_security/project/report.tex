\documentclass[a4paper,10pt,final,fleqn]{article}
\usepackage[frenchb]{babel}
\usepackage{fontenc}
\usepackage{fancyhdr} % Required for custom headers
\usepackage{lastpage} % Required to determine the last page for the footer
\usepackage{extramarks} % Required for headers and footers
\usepackage{graphicx} % Required to insert images
\usepackage[utf8]{inputenc}
\usepackage{apacite}
\usepackage{url}
\usepackage[normalem]{ulem}
\usepackage{verbatim}
\usepackage{listings}

\evensidemargin=0in
\oddsidemargin=0in
\textwidth=6in
\textheight=9.0in
\headsep=0.25in 

% Set up the header and footer
\pagestyle{fancy}
\lhead{\GroupeName} % Top left header
\chead{\CourseAndAPP} % Top center header
\rhead{\Date} % Top right header
\lfoot{\lastxmark} % Bottom left footer
\cfoot{} % Bottom center footer
\rfoot{\ \thepage\ / 	\pageref{LastPage}} % Bottom right footer
\renewcommand\headrulewidth{0.4pt} % Size of the header rule
\renewcommand\footrulewidth{0.4pt} % Size of the footer rule

\setcounter{tocdepth}{2}

\newcommand{\GroupeName}{HE201041}
\newcommand{\CourseAndAPP}{BeerCollection} % Course/APP
\newcommand{\Date}{\today}

\title{
\parbox{15cm}
{ %\includegraphics[width=4cm]{foxhound.png} \\
  \vspace{3cm}
	\begin{center}\sf\bfseries\Huge
		\rule{15cm}{1pt}
		\medskip
		Sécurité des Réseaux \\
		\huge Projet IpTables
		\vspace{.5cm}
		\rule{15cm}{1pt}
	\end{center}
	\vspace{3cm}
 }} 
\author{Monroe Samuel}
\date{\today}

\begin{document}
\maketitle
\newpage
\thispagestyle{empty}
\mbox{}

\section{Introduction}
	
	Ce rapport fait état de notre travail sur le projet Iptables dans le cadre du cours de Sécurité des Réseaux donné par Madame Van Den Schrieck.\\

	Il contiendra une explication des règles haut niveau implémentées, et notre stratégie de validation ainsi que leurs résultats.\\
	En annexe seront également incluses les configurations des trois firewalls, ainsi que les scripts ou méthodologies de validation.\\

\section{Pré-configurations et Environnement}

	Tout d'abord, ce projet a été réalisé via le labo Netkit fournit par Mme. Van Den Schrieck, tournant dans un environnement Centos 6 Securité virtualisé sous VmWare.\\

	Ceci a un impact au niveau de la configuration initiale, le labo Netkit n'avait tout simplement pas accès à internet. Quelques petits points de configurations sont donc nécéssaires pour poursuivre l'implémentation de notre solution : \\

	\begin{enumerate}
		\item Dans la machine virtuelle, il faut déterminer sur quelle interface est liée le labo Netkit, elle s'apelle typiquement \textbf{nk_tap_user}.\\
		\item Il faut également déterminer quelle adresse ip est assignée à cette interface, ici \textbf{10.0.1.15}\\
		\item Dans le \textbf{router} du labo Netkit, il faut définir sa route par défaut (gateway) sur l'adresse ip de l'interface nk_tap_user : \textbf{route add default gw 10.0.1.15}.\\
		\item Il faut maintenant activer le NAT sur la Vm pour que le tout fonctionne, en déterminant au préalable sous quelle interface la VM elle-même est reliée à l'Internet (eth6): \\
		\begin{itemize}
			\item \textbf{iptables -t nat -I POSTROUTING 1 -o eth6 -j MASQUERADE}
			\item \textbf{iptables -I FORWARD 1 -i nk_tap_user -j ACCEPT}
			\item \textbf{echo "1" > /proc/sys/net/ipv4/ip_forward}
		\end{itemize}

		\item ????
		\item PROFIT !!!
	\end{enumerate}

	Un ping 8.8.8.8 depuis router devrait maintenant fonctionner, et une implémentation des règles NAT au sein du réseau et des règles Firewall peut être commencée.\\

\end{document} 
