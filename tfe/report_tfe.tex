\documentclass{report}
\usepackage{filecontents}

\usepackage[utf8]{inputenc}
\usepackage[T1]{fontenc}
\usepackage[francais]{babel}
\usepackage{listings}
\usepackage[a4paper]{geometry}
\usepackage{graphicx}
\usepackage[export]{adjustbox}
\usepackage{titlesec}
\usepackage{color}
\usepackage[toc, page]{appendix}
\usepackage{url}
\usepackage[left=2cm,top=2.5cm,right=1.5,bottom=2.5cm]{geometry}

\definecolor{xcodekw}{rgb}{0.75, 0.22, 0.60}
\definecolor{xcodestr}{rgb}{0.89, 0.27, 0.30}
\definecolor{xcodecmt}{rgb}{0.31, 0.73, 0.35}

% \titleformat{\chapter}[display]
%   {\centering\normalfont\huge\bfseries}
%   {\chaptertitlename\ \thechapter}
%   {20pt}
%   {\Huge}

\geometry{hscale=0.75,vscale=0.85,centering}

\renewcommand{\thesection}{\arabic{section}}
\renewcommand\appendixtocname{Annexes}
\renewcommand\appendixname{Annexes}
\renewcommand\appendixpagename{Annexes}


\begin{document}

\begin{titlepage}

\newcommand{\HRule}{\rule{\linewidth}{0.5mm}} % Defines a new command for the horizontal lines, change thickness here

\center % Center everything on the page


\textsc{\LARGE Ecole Pratique des Hautes Etudes Commerciales }\\[1.5cm] % Name of your university/college
\includegraphics[scale=0.7]{ephec.png}\\
Avenue du Ciseau, 15\\
1348 Louvain-la-neuve\\
\\[50pt]
\textsc{\Large Travail de fin d'études}\\[0.5cm] % Major heading such as course name


\HRule \\[0.4cm]
{ \huge \bfseries Plateforme d'apprentissage de l'anglais via chat et visioconférence}\\[0.4cm] % Title of your document
\HRule \\[1.5cm]

\textsc{Travail de fin d’études présenté en vue de l’obtention du diplôme de bachelier en Informatique et Systèmes : finalité Technologie de l’informatique}\\

\\[90pt]

\begin{minipage}{0.4\textwidth}
\begin{flushleft} \large
\emph{Auteur:}\\
Samuen \textsc{Monroe} 3TL1 % Your name
\end{flushleft}
\end{minipage}
~
\begin{minipage}{0.4\textwidth}
\begin{flushright} \large
\emph{Rapporteur (ephec) :} \\
Virginie \textsc{Van den Schrieck} % Supervisor's Name
\end{flushright}
\end{minipage}\\[4cm]



{\large \today}\\[3cm] % Date, change the \today to a set date if you want to be precise


\end{titlepage}

\newpage
\thispagestyle{empty}
\mbox{}

\tableofcontents
\clearpage

%% \textbf{}

\chapter*{Remerciements}

  Je voudrais tout d'abord remercier Virginie Van Den Schrieck qui, en tant que rapporteuse de mon travail de fin d'études,
  m'a accompagné et aidé à parvenir au bout du projet que je me suis fixé pour ce travail.
  Merci à elle mais également à Marie-Noël Vroman pour le rôle qu'elles ont tenu lors du projet d'intégration du premier quadrimestre,
  qui m'a fait entrer dans le développement Ruby on Rails et a fait naître en moi la volonté de poursuivre une carrière dans le développement web.

  Je remercie Nicolas Jacobeus, CEO de chez Belighted, de m'avoir offert cette place de stage pendant ces quinze
  semaines où j'ai énormément progressé et pu apprendre de ce métier que je veux faire plus tard, ainsi que tous les membres
  de l'équipe pour leur accueil, leurs conseils et leur aide tout au long du stage, et plus spécifiquement Maxime Nguyen et Michaël Albert qui m'ont
  offert de précieux conseils sur AngularJs et WebRTC.

  Je voudrais également remercier Yves Delvigne, qui par son cours de programmation multimédia
  nous a inculqué les bases du développement web, et qui a donc grandement contribué
  au chemin que j'ai entrepris de suivre depuis la deuxième année.

  Un merci particulier à Rémy Voet, avec qui nous nous sommes motivés mutuellement en se retrouvant pour
  travailler sur nos projets respectifs.

  Mes remerciements vont enfin à tous les autres professeurs de l'EPHEC qui m'ont accompagné tout au long
  de ce cursus de bachelier en technologie de l'informatique et qui ont, quelle que soit la matière
  enseignée, contribué à me transmettre un panel de connaissances varié qui m'est et me sera utile pour
  le reste de ma vie. Merci donc à Christian Lambeau, Claude Masson, Michel de Vleeschouwer,Youcef Bouterfa, Arnaud Dewulf,
  Stéphane Faulkner, Véronique Leclerq, Thomas Thiry, Maxime Vanlerberghe, Jean-François Depasse, Marc Deherve et Laurent Hirsoux.


\chapter{Introduction}

  Conjoitement au stage en entreprise, le TFE marque la fin de ces trois années de bachelier en technologie de l'informatique.
  C'est un travail de longue haleine, prenant l'entiereté du deuxième quadrimestre, associé au devoir de gérer son temps par rapport
  à celui investi dans chaque journée au stage.
  Mais c'est aussi l'occasion de pouvoir mettre en oeuvre nos compétences via la création d'un produit conçu enièrement par nos soins.

  Passionné par le développement web, j'ai décidé de me lancer dans la création d'une application web destinée à faciliter l'apprentissage de l'anglais
  en permettant aux utilisateurs de communiquer entre eux dans ce but grâce à un chat textuel en temps réel et à la possibilité de communiquer via un chat
  audio/vidéo.

\chapter{Conclusion}
\end{document}
