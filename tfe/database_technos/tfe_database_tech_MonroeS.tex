\documentclass[a4paper,10pt,final,fleqn]{article}
\usepackage[frenchb]{babel}
\usepackage{fontenc}
\usepackage{fancyhdr} % Required for custom headers
\usepackage{lastpage} % Required to determine the last page for the footer
\usepackage{extramarks} % Required for headers and footers
\usepackage{graphicx} % Required to insert images
\usepackage[utf8]{inputenc}
\usepackage{apacite}
\usepackage{url}
\usepackage[normalem]{ulem}
\usepackage{verbatim}
\usepackage{listings}

\evensidemargin=0in
\oddsidemargin=0in
\textwidth=6in
\textheight=9.0in
\headsep=0.25in 

% Set up the header and footer
\pagestyle{fancy}
\lhead{\GroupeName} % Top left header
\chead{\CourseAndAPP} % Top center header
\rhead{\Date} % Top right header
\lfoot{\lastxmark} % Bottom left footer
\cfoot{} % Bottom center footer
\rfoot{\ \thepage\ / 	\pageref{LastPage}} % Bottom right footer
\renewcommand\headrulewidth{0.4pt} % Size of the header rule
\renewcommand\footrulewidth{0.4pt} % Size of the footer rule

\setcounter{tocdepth}{2}

\newcommand{\GroupeName}{HE201041}
\newcommand{\CourseAndAPP}{TFE} % Course/APP
\newcommand{\Date}{\today}

\title{
\parbox{15cm}
{ %\includegraphics[width=4cm]{foxhound.png} \\
  \vspace{3cm}
	\begin{center}\sf\bfseries\Huge
		\rule{15cm}{1pt}
		\medskip
		TFE \\
		\huge Modèles, Réseau et Choix Techniques
		\vspace{.5cm}
		\rule{15cm}{1pt}
	\end{center}
	\vspace{3cm}
 }} 
\author{Monroe Samuel}
\date{\today}

\begin{document}
\maketitle
\newpage

	\section{Introduction}

		Ce document consiste en une construction d'une base de modèle de données, la proposition d'un schéma réseau et référencera les choix techniques que je souhaite poser.\\

	\section{Modèles de données}

		Le modèle ici présenté est incomplet uniquement quant aux différentes informations personnelles sortant du cadre de la table User, mais qui typiquement dans une base de données relationnelle vont s'exprimer toujours de la même manière que \textbf{Hobbies} et \textbf{Languages}.\\
		J'ai donc ici omis de surcharger le schéma de structures redondantes.\\

		\includegraphics[scale=0.38]{database.png}\\

	\section{Réseau}

		Cette partie est assez simple, étant donné que je souhaite louer un VPS chez DigitalOcean et y faire tourner les services suivants : \\

		\begin{itemize}
			\item Nginx pour le Web et services associés
			\item PostgreSQL
			\item Signalling Server pour le WebRTC\\
		\end{itemize}

	\section{Choix techniques}

		La première contrainte ayant été identifiée dans le document précédent était l'aspect real-time que demande ce genre de service, ceci est faisable en utilisant \textbf{Pusher} avec mon stack Rails/Angular et le client Javascript de l'application mobile.\\

		Deuxièmement je pointais le problème audio/vidéo et du possible environnement VoIP à intégrer dans l'infrastructure, \textbf{Pusher} me permettrait également en combinaison de \textbf{WebRTC}, d'obtenir une solution fonctionnelle d'échange vidéo/audio en temps réel.\\

		Enfin, je pointais l'API nécéssaire à l'app mobile, elle sera fournie par mon backend \textbf{Rails}.\\

		Pour résumer, voici une liste des différents outils répertoriés à ce jour : \\
		
		\begin{itemize}
			\item \textbf{Langages} : Ruby on Rails, Javascript(Angular), HTML, CSS
			\item \textbf{Servers} : PostgreSQL, Nginx, Ubuntu
			\item \textbf{Autres} : PhoneGap, Pusher, WebRTC, Framwork de design du type Bootstrap\\
		\end{itemize}

\end{document}