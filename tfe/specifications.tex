\documentclass[a4paper,10pt,final,fleqn]{article}
\usepackage[frenchb]{babel}
\usepackage{fontenc}
\usepackage{fancyhdr} % Required for custom headers
\usepackage{lastpage} % Required to determine the last page for the footer
\usepackage{extramarks} % Required for headers and footers
\usepackage{graphicx} % Required to insert images
\usepackage[utf8]{inputenc}
\usepackage{apacite}
\usepackage{url}
\usepackage[normalem]{ulem}
\usepackage{verbatim}
\usepackage{listings}

\evensidemargin=0in
\oddsidemargin=0in
\textwidth=6in
\textheight=9.0in
\headsep=0.25in 

% Set up the header and footer
\pagestyle{fancy}
\lhead{\GroupeName} % Top left header
\chead{\CourseAndAPP} % Top center header
\rhead{\Date} % Top right header
\lfoot{\lastxmark} % Bottom left footer
\cfoot{} % Bottom center footer
\rfoot{\ \thepage\ / 	\pageref{LastPage}} % Bottom right footer
\renewcommand\headrulewidth{0.4pt} % Size of the header rule
\renewcommand\footrulewidth{0.4pt} % Size of the footer rule

\setcounter{tocdepth}{2}

\newcommand{\GroupeName}{HE201041}
\newcommand{\CourseAndAPP}{TFE} % Course/APP
\newcommand{\Date}{\today}

\title{
\parbox{15cm}
{ %\includegraphics[width=4cm]{foxhound.png} \\
  \vspace{3cm}
	\begin{center}\sf\bfseries\Huge
		\rule{15cm}{1pt}
		\medskip
		TFE \\
		\huge Spécifications
		\vspace{.5cm}
		\rule{15cm}{1pt}
	\end{center}
	\vspace{3cm}
 }} 
\author{Monroe Samuel}
\date{\today}

\begin{document}
\maketitle
\newpage

	\section{Rappel}

		La première étape indispensable dans la réalisation de ce travail est de définir un cahier des charges, c’est-à-dire une description non technique (point de vue utilisateur) de la réalisation que vous envisagez.

		Je vous propose donc de vous lancer dans la rédaction d’une première version de ce document (environ 2-3 pages).  Ce document devrait lister l’ensemble des fonctionnalités du projet, les contraintes qui s’y appliquent éventuellement (ces dernières peuvent, elles, être d’ordre technique) et fournir une première description générale des interfaces (graphiques ou non) permettant l’interaction avec l’utilisateur.

	\section{Introduction}

		Ce document consiste en un cahier des charges d'un point de vue utilisateur, le référencement des fonctionnalités que le projet comportera ainsi que les contraintes possibles qui s'y appliquent.\\

		Je commencerai par rappeler brièvement en quoi consiste le projet, et poursuivrai avec une description de chaque fonctionnalité en imaginant le parcours d'une personne désirant utiliser le produit.\\

	\section{Rappel du projet}

		L'idée est de développer une webapp d'apprentissage de l'anglais par correspondance via VoIP et VidéoChat intégrés sur cette application, une application mobile serait également disponible conjointement au site internet.\\

		Un utilisateur se verrait proposer, sur base d'informations personnelles et sur ses centres d'intérêts, une liste d'autres personnes ayant des points communs avec celle-ci et de plus, n'ayant \textbf{aucune} langue commune autre que l'anglais, de sorte que celui-ci soit l'unique moyen de communiquer entre eux.\\
		Le service s'adressant à des personnes désireuses d'apprendre l'anglais, ce seul moyen de communication les ferait progresser inévitablement dans la langue, ne serait-ce quand s'habituant à écrire des choses courantes en anglais, et mieux encore en conversant via le chat vocal-vidéo.\\

	\section{Cahier des Charges}

		Ici vont se succéder les sections concernant chacune des fonctionnalités, listées de manière à simuler le chemin que prendrait un nouvel utilisateur du service, en séparant les fonctionnalités selon le support Web ou Application Smartphone.\\

		\subsection{Web App}

			\subsubsection{Page d'accueil}

				En tant qu'utilisateur, je veux pouvoir avoir une vue d'ensemble de ce que propose le service, et comprendre clairement les fonctionnalités que celui-ci me propose, afin d'avoir envie d'utiliser celui-ci si il peut me fournir ce que je suis venu chercher.\\
				Il faut également que je puisse savoir où m'enregistrer afin de commencer mon expérience, ou me connecter si je fais déjà partie de la communauté.\\

				Pour ce faire, j'imagine une page d'accueil du type de Facebook

	\section{Détails Techniques Additionnels}

		Vous trouverez dans cette section des éléments non abordés précédemment mais que j'estime intéressant de répertorier, puisqu'ils peuvent être sujet à discussion et à modification.\\
		Je décrirai notamment les langages, technologies, infrastructures que je compte utiliser pour ce projet.\\

		Langages (frameworks) utilisés : \\

		\begin{itemize}
			\item \textbf{Ruby on Rails} : Ayant déjà une expérience sur celui-ci pour intégration, et étant donné que je vais également utiliser celui-ci lors de mon stage, je pense que c'est le meilleur choix qui puisse s'offrir à moi.\\
			De plus, une bonne RESTapi sera pratique à mettre en place pour l'application mobile.\\

			 \item \textbf{}
		\end{itemize}


\end{document}