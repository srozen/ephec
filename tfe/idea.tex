### IDEA 1

Création d'une plateforme web d'apprentissage de l'anglais par correspondance, via VoIP et VidéoChat intégrée à la plateforme.

Le but de l'application sera de proposer à un utilisateur, sur base de centres d'intérêts qu'il aura indiqué à l'inscription, d'être en relation
avec d'autres gens provenant obligatoirement d'un pays étranger, et dans l'idéal ne partageant aucune langue "connue" autre que l'anglais avec cet utilisateur.
Cette relation aura donc pour intérêt de ne proposer que l'anglais comme moyen de communication entre les deux personnes, et donc faire progresser 
leurs compétences dans cette langue.

Une application mobile multi-plateforme sera également disponible pour les utilisateurs, afin de pouvoir garder contact avec leurs correspondants, 
et accéder en tout temps aux services proposés.

Au niveau des technologies, je compte utiliser Node.js pour le back-end, et je dois encore analyser et déterminer quel framework front-end utiliser.
Une infrastructure Unix sera évidemment nécéssaire pour supporter le service, avec WebRTC pour les communications vocales et vidéos.
Au niveau de l'app mobile, je pense la développer en utilisant PhoneGap.

### IDEA 2 

Plateforme Web dédiée à la collaboration pour les cours.

Le but de l'application serait de proposer aux étudiants d'une même classe de créer leur groupe de cours sur cette plate-forme.
A partir de ce moment, ils ont accès à une plate-forme de communication qui centralise leur échanges, et permet d'éviter une solution bancale telle que Facebook pour ces échanges.
Cet espace leur permettretrait des créer une sous-section destinée à chacun des cours, et pour chaque cours (ou dans un cadre global) un espace de stockage de fichiers concernant le cours (ou l'année scolaire).

L'aspect collaboratif résiderait dans le fait que pour chaque cours, un ou plusieurs fichiers éditables par le groupe seraient présents, et rédigeables en Markdown, afin d'élaborer des synthèses complètes constituées des notes de tout les étudiants.
Un système serait prévu pour versionner ou sauvegarder les états de ces synthèses (vote sur les modifications), afin d'éviter des suppressions sauvages ou tout autre abus.