\documentclass{report}
\usepackage{filecontents}

\usepackage[utf8]{inputenc}
\usepackage[T1]{fontenc}
\usepackage[francais]{babel}
\usepackage{listings}
\usepackage[a4paper]{geometry}
\usepackage{graphicx}
\usepackage[export]{adjustbox}
\usepackage{titlesec}
\usepackage{color}
\usepackage[toc, page]{appendix}
\usepackage{url}

\definecolor{xcodekw}{rgb}{0.75, 0.22, 0.60}
\definecolor{xcodestr}{rgb}{0.89, 0.27, 0.30}
\definecolor{xcodecmt}{rgb}{0.31, 0.73, 0.35}

\titleformat{\chapter}[display]
  {\centering\normalfont\huge\bfseries}
  {\chaptertitlename\ \thechapter}
  {20pt}
  {\Huge}

\geometry{hscale=0.75,vscale=0.85,centering}

\renewcommand{\thesection}{\arabic{section}}
\renewcommand\appendixtocname{Annexes}
\renewcommand\appendixname{Annexes}
\renewcommand\appendixpagename{Annexes}

\title{Rapport de Stage}
\author{Samuel \bsc{Monroe}}

\date{\today}

\begin{document}

\maketitle

\newpage
\thispagestyle{empty}
\mbox{}

\tableofcontents

\chapter*{Remerciements}

  Je voudrais tout d'abord remercier Virginie Van Den Schrieck et Marie-Noël Vroman, pour
  le rôle qu'elles ont tenu dans le projet d'intégration du premier quadrimestre,
  en nous ayant responsabilisé quant à la gestion de notre projet et donné carte
  blanche pour la réalisation de celui-ci.
  C'est donc grâce à elles que j'ai fais mes premiers pas dans le monde du Ruby on Rails
  et du développement web moderne en général, et ainsi avoir eu l'opportunité de faire ce stage chez Belighted.\\

  Je remercie Nicolas Jacobeus, CEO de chez Belighted, de m'avoir offert cette place de stage pendant ces quinze
  semaines où j'ai énormément progressé et appris de ce métier que je veux faire plus tard, ainsi que tout les membres
  de l'équipe pour leur accueil, leurs conseils et leur aide tout au lond du stage.\\

  Je voudrais également remercier Yves Delvigne, qui par son cours de programmation multimédia
  nous a inculqué les bases du développement web, et qui a donc grandement contribué
  au chemin que j'ai suivi depuis la deuxième année.\\

  Mes remerciements vont enfin à tout les autres professeurs de l'EPHEC qui m'ont accompagnés tout au long
  de ce cursus de bachelier en technologies de l'informatique et qui ont, quel que soit la matière
  enseignée, contribués à me transmettre un panel de connaissances varié qui m'est et me sera utile pour
  le reste de ma vie. Merci donc à Christian Lambeau, Claude Masson, Michel De Vleeschouwer,Youcef Bouterfa, Arnaud Dewulf,
  Stéphane Faulkner, Véronique Leclerq, Thomas Thiry, Maxime Vanlerberghe, Jean-François Depasse, Marc Deherve et Laurent Hirsoux.\\

\chapter{Introduction}

  Point d'orgue de ces trois années de bachelier en technologie de l'informatique, la perspective de ce stage en entreprise
  est en même temps excitante, motivante mais aussi un peu effrayante.\\
  C'est le moment où nos acquis académiques sont confrontés à la réalité du terrain, celui du premier vrai pas vers la sortie
  de classes de l'ephec.\\

  C'est en direction de Belighted que j'ai fait ce pas, qui m'a accueilli grâce à une petite expérience déjà acquise en Ruby on Rails.\\
  Lors de nos entretiens avant le stage, l'objectif et le travail que j'allais accomplir n'était pas clairement défini, néamoins il était
  très clair qu'ils seraient de contribuer à un ou plusieurs projets au sein de l'équipe en faisant évoluer les compétences que j'avais acquises.\\
  L'objectif qu'on m'a confié lors du premier jour de stage, que je décrirai plus en détails dans les pages suivantes, était de mettre sur pied un Software as a Service qui offrirait une gestion des avatars très simple
  à d'autres applications web, leur évitant tout besoin de mettre en place un espace de stockage dédié et d'implémenter un
  système d'avatars.\\

  Concernant mes attentes personnelles pour ce stage, j'étais en premier lieu motivé par la perspective d'apprendre un maximum
  sur le développement web. Je voulais vivre une expérience dans une PME travaillant dans le domaine, découvrir quelles étaient les bonnes
  pratiques suivies lors du développement, quelles étaient les étapes du cycle de vie d'une application web.\\
  Plus encore, je voulais avoir l'opportunité d'être très content de me lever chaque matin pour aller faire quelques chose d'agréable et de passionant
  dans un cadre lui-même agréable.\\


\chapter{A propos de Belighted}

  Belighted est une PME active depuis 2008 sur le marché du developpement d'applications web et mobiles.\\
  L'équipe est composée d'un peu de dix personnes, jeunes et passionnées, toutes travaillant dans le même open-space. Tout ceci donne
  une ambiance sûrement plus proche de la startup que de l'entreprise classique.\\

  L'entreprise s'est spécialisée dans le développement en Ruby on Rails, faisant d'elle et de son équipe des experts et la plus grosse
  boîte de développement en Rails de Belgique.\\
  Cette expertise se fonde sur une recherche de la qualité, en suivant notamment des principes Agiles.

\chapter{Nature du stage et modalités de réalisation}

\chapter{Objets d'études / Faits saillants / Expériences vécues}

\chapter{Conclusion}

\end{document}
